%
% The first command in your LaTeX source must be the \documentclass command.
\documentclass[sigplan,screen]{acmart}

%
% defining the \BibTeX command - from Oren Patashnik's original BibTeX documentation.
\def\BibTeX{{\rm B\kern-.05em{\sc i\kern-.025em b}\kern-.08emT\kern-.1667em\lower.7ex\hbox{E}\kern-.125emX}}
    
% Rights management information. 
% This information is sent to you when you complete the rights form.
% These commands have SAMPLE values in them; it is your responsibility as an author to replace
% the commands and values with those provided to you when you complete the rights form.
%
% These commands are for a PROCEEDINGS abstract or paper.
\copyrightyear{2018}
\acmYear{2018}
\setcopyright{acmlicensed}
\acmConference[Woodstock '18]{Woodstock '18: ACM Symposium on Neural Gaze Detection}{June 03--05, 2018}{Woodstock, NY}
\acmBooktitle{Woodstock '18: ACM Symposium on Neural Gaze Detection, June 03--05, 2018, Woodstock, NY}
\acmPrice{15.00}
\acmDOI{10.1145/1122445.1122456}
\acmISBN{978-1-4503-9999-9/18/06}

%
% These commands are for a JOURNAL article.
%\setcopyright{acmcopyright}
%\acmJournal{TOG}
%\acmYear{2018}\acmVolume{37}\acmNumber{4}\acmArticle{111}\acmMonth{8}
%\acmDOI{10.1145/1122445.1122456}

%
% Submission ID. 
% Use this when submitting an article to a sponsored event. You'll receive a unique submission ID from the organizers
% of the event, and this ID should be used as the parameter to this command.
%\acmSubmissionID{123-A56-BU3}

%
% The majority of ACM publications use numbered citations and references. If you are preparing content for an event
% sponsored by ACM SIGGRAPH, you must use the "author year" style of citations and references. Uncommenting
% the next command will enable that style.
%\citestyle{acmauthoryear}

%
% end of the preamble, start of the body of the document source.
\begin{document}

%
% The "title" command has an optional parameter, allowing the author to define a "short title" to be used in page headers.
\title{Enabling Agile Through Teamwork}

%
% The "author" command and its associated commands are used to define the authors and their affiliations.
% Of note is the shared affiliation of the first two authors, and the "authornote" and "authornotemark" commands
% used to denote shared contribution to the research.
\author{Cecilia La Place}
\authornote{Both authors contributed equally to this research.}
\email{claplace@asu.edu}
\orcid{1234-5678-9012}
\author{Chiranjeevi Ramamurthy}
\authornotemark[1]
\email{cramamu1@asu.edu}
\author{Alexandra Mehlhase}
\email{a.mehlhase@asu.edu}
\affiliation{%
  \institution{Arizona State University}
  \streetaddress{7001 E Williams Field Rd}
  \city{Mesa}
  \state{Arizona}
  \postcode{85212}
}

%
% By default, the full list of authors will be used in the page headers. Often, this list is too long, and will overlap
% other information printed in the page headers. This command allows the author to define a more concise list
% of authors' names for this purpose.
\renewcommand{\shortauthors}{La Place and Ramamurthy}

%
% The abstract is a short summary of the work to be presented in the article.
\begin{abstract}
For more than a decade, the Agile development process has seeped into the lives of software developers and customers, changing the way projects are planned, how teammembers and teams interact, and how customers receive their product. Agile is a teamwork heavy process, demanding superb communication and technical skills to develop projects where requirements can change the flow of the project between sprints. Similiarly, multi-team agile projects need to consider team management and team communication, as well as using distinct architectures and designs. 
\end{abstract}

%
% The code below is generated by the tool at http://dl.acm.org/ccs.cfm.
% Please copy and paste the code instead of the example below.
%
%\begin{CCSXML}
%<ccs2012>
% <concept>
%  <concept_id>10010520.10010553.10010562</concept_id>
%  <concept_desc>Computer systems organization~Embedded systems</concept_desc>
%  <concept_significance>500</concept_significance>
% </concept>
% <concept>
%  <concept_id>10010520.10010575.10010755</concept_id>
%  <concept_desc>Computer systems organization~Redundancy</concept_desc>
%  <concept_significance>300</concept_significance>
% </concept>
% <concept>
%  <concept_id>10010520.10010553.10010554</concept_id>
%  <concept_desc>Computer systems organization~Robotics</concept_desc>
%  <concept_significance>100</concept_significance>
% </concept>
% <concept>
%  <concept_id>10003033.10003083.10003095</concept_id>
%  <concept_desc>Networks~Network reliability</concept_desc>
%  <concept_significance>100</concept_significance>
% </concept>
%</ccs2012>
%\end{CCSXML}

%\ccsdesc[500]{Computer systems organization~Embedded systems}
%\ccsdesc[300]{Computer systems organization~Redundancy}
%\ccsdesc{Computer systems organization~Robotics}
%\ccsdesc[100]{Networks~Network reliability}

%
% Keywords. The author(s) should pick words that accurately describe the work being
% presented. Separate the keywords with commas.
\keywords{agile, teamwork, design, architecture, software development process}

%
% A "teaser" image appears between the author and affiliation information and the body 
% of the document, and typically spans the page. 
%\begin{teaserfigure}
%  \includegraphics[width=\textwidth]{sampleteaser}
%  \caption{Seattle Mariners at Spring Training, 2010.}
%  \Description{Enjoying the baseball game from the third-base seats. Ichiro Suzuki preparing to bat.}
%  \label{fig:teaser}
%\end{teaserfigure}

%
% This command processes the author and affiliation and title information and builds
% the first part of the formatted document.
\maketitle

\section{Introduction}
Agile development considers an element crucial to all software processes in a uniquely different manner: customers as team members. The basis of all software development is planning out a project from start to end, but requirements are prone to change when considering the fluctuating world of technology. Agile handles change by valuing the people involved, customers and developers \cite{Highsmith01}. By interfacing with customers frequently, and considering their changing requests as the project progresses, change is a byproduct. However, the success of Agile is reliant on what Cockburn and Highsmith call "responsive people and organizations" as well as "[focusing] on the talents and skills of individuals" \cite{Cockburn01}. As a result, teamwork and communication become unavoidably ensconsed in the agile process.
In this work, we first, introduce a quick overview of Agile and agile teams in order to discuss multi-team projects in agile. Second, we discuss the different architectures used alongside these projects and design considerations.

\section{Teams in Agile}
A single team in agile consists of the product owner, the scrum master, and the developers. In order for their team to be successful, they must "have a common focus, mutual trust, and respect," be "collarborative, but speedy [in their] decision-making process" and be adept at handling ambiguity \cite{Cockburn01}. The Agile Manifesto reinforces these attributes by valuing customer collaboration and responding to change over contracts and plans \cite{Beck01}. Daily stand-ups incite communication, updating each other on task statuses, and conveying problems if they exist. Depending on the needs that follow the stand-up, the team will respond accordingly. e.g, team members may pair up to get a difficult or time sensitive task done.

\subsection{Scaling Up Teams in Agile}
As projects scale up, so does the number of teams. When considering this in an agile workspace, there is more work that must go into ensuring that teams themselves can collaborate effectively. If teams are in the same location, then promoting a collaborative environment is key. However, teams across more than one location must account for different cultures and providing as much possibility for communication as possible \cite{Joshi12}. Distance increases the potential for misunderstandings. Other recommendations include having a single product backlog \cite{Zalavadia16}, cross-team daily stand-ups (and other cross-team activities) \cite{GSATech}, and having communities of practice \cite{Crocker18}. Each of these suggestions tackles a unique but very real and potential problem that occurs in large scale agile practices. Having a single product backlog allows all teams to see the whole picture, and see where work is needed. Cross-team activities allows for a deeper understanding of the project and each team's accomplishments and troubles. Finally, communities of practices creates a valuable way to share information about relevant topics from other perspectives.

\section{Design and Architecture in Agile}
\subsection{Architecture}

Organizations tend to fail if their agile process is not bounded by a fail-safe plan.

Constraints such as financial, regulatory, technical or customer driven becomes a fundamental reason for the very existence of this issue. There is a new emerging trend, which explores the possibility of using Agile practices in order to manage traditional business functions. \cite{SoftArchAgile}

The SAFe (Scaled Agile Framework) introduces two distinct elements of Architecture in SAFe:
\begin{itemize}
\item {\verb|Emergent Design|}
\item {\verb|Intentional Architecture|}
\end{itemize}

Technical basis for development and the incremental implementation of initiatives are provided by \textit{Emergent Design}. In order to ensure the initiative continually delivers value, Emergent Design helps the Designers and Architects to be highly responsive to ever-changing customers/ stakeholders. At this juncture, Architecture is SAFe can be seen as a collaborative and interactive exercise through which the design element can emerge. 

\textit{Intentional Architecture} is a more traditional Architecture where the performance and usability of the initiative is supported and enhanced by a set of well-defined and planned Architectural initiatives. The severity and importance of the constraints such as choice of technology platform, financial budget, etc is clearly visualized using this Intentional Architecture.
The probability of the initiative being successful and delivering value is increased if these constraints can be identified and incorporated into the initiative.

The key to the success here is the level of abstraction at which the balance of Emergent Design and Intentional Architecture occur. The fundamental behavior that will determine this is collaboration. 

Agile requires Architecture that supports how Agile Practices deliver outcomes (value). This is achieved through a combination of a nimble reactive style of Architecture supported by a more traditional structured approach to Architecture \cite{EnterpriseArch}
\subsection{Design}
Given the fact that design is an important part of a software project, yet the development team usually struggle for coming up with a right design. This happen due to many reasons such as, focusing on high-fidelity design, which in-turn forces the project to adapt waterfall approach, etc.,

Effectively integrating design into agile process is major software developers' struggle. There is potential danger that extra work could be generated if a designer doesn't work closely with their team. This could lead to a creation of harmful silos of knowledge within the team.\cite{AgileDesignPrac}

In order to avoid this a proper design is required, especially when a cross functional team is involved in developing a product for the intended customers/shareholders. 

This can be effectively done when designers are included in agile planning process and paired with a developer to share their knowledge. This can be further improvised when all the work flow is made visible across entire project. This can be ensured by having a demo or a prototype display periodically, so that every one in the team can have an idea. Having this approach enables a collaborative view, rather than a traditional linear view.

\section{Conclusion}
In conclusion, agile projects with multiple teams are heavily dependent on communication, collaborative cultures, and extra information resources. As with team scaling, architecture must scale too, in order to promote the success of the project. [Design conclusion]

%
% The next two lines define the bibliography style to be used, and the bibliography file.
\bibliographystyle{ACM-Reference-Format}
\bibliography{sample-base}

\end{document}